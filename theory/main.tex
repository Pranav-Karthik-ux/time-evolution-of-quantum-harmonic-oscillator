\documentclass{article}
\usepackage{amsmath}
\usepackage{amssymb}
\usepackage{physics}

\title{Time Evolution of the Quantum Harmonic Oscillator}
\author{}
\date{}

\begin{document}
\maketitle{}
\section{The Quantum Harmonic Oscillator}

The Hamiltonian of the one-dimensional quantum harmonic oscillator is
\begin{align}
\hat{H} = \frac{\hat{p}^2}{2m} + \frac{1}{2} m \omega^2 \hat{x}^2 = \hbar \omega \left( \hat{a}^\dagger \hat{a} + \frac{1}{2} \right),
\end{align}
where $\hat{a}$ and $\hat{a}^\dagger$ are the lowering and raising operators:
\begin{align}
\hat{a} = \sqrt{\frac{m\omega}{2\hbar}} \left( \hat{x} + \frac{i}{m\omega} \hat{p} \right), \quad
\hat{a}^\dagger = \sqrt{\frac{m\omega}{2\hbar}} \left( \hat{x} - \frac{i}{m\omega} \hat{p} \right).
\end{align}

The energy eigenvalues and eigenstates (number states) are
\begin{align}
\hat{H} \ket{n} &= E_n \ket{n}, \qquad n = 0,1,2,\dots \\
E_n &= \hbar \omega \left( n + \frac{1}{2} \right).
\end{align}

\section{Position-Space Wave Functions}

The normalized energy eigenfunctions in position representation are
\begin{align}
\psi_n(x) &= \bra{n}x\ket{n} = \frac{1}{\sqrt{2^n n!}} \left( \frac{m\omega}{\pi \hbar} \right)^{1/4} 
H_n(\xi) e^{-\xi^2/2},
\end{align}
where $\xi = \sqrt{\frac{m\omega}{\hbar}} x$ is the dimensionless coordinate, and $H_n(\xi)$ are the (physicists’) Hermite polynomials:
\begin{align}
H_n(\xi) = (-1)^n e^{\xi^2} \frac{d^n}{d\xi^n} e^{-\xi^2}.
\end{align}

It is common to define the characteristic length
\begin{align}
x_0 = \sqrt{\frac{\hbar}{m\omega}},
\end{align}
so that $\xi = x/x_0$ and the wave functions can be written as
\begin{align}
\psi_n(x) = \frac{1}{\sqrt{2^n n! \, x_0 \sqrt{\pi}}} \, H_n\!\left( \frac{x}{x_0} \right) 
\exp\!\left( -\frac{x^2}{2x_0^2} \right).
\end{align}

\section{Time Evolution of Energy Eigenstates}

Since $\ket{n}$ are energy eigenstates, their time evolution is simply a phase factor:
\begin{align}
\ket{n(t)} &= e^{-i E_n t / \hbar} \ket{n(0)} = e^{-i \omega (n + 1/2) t} \ket{n}.
\end{align}
Thus, in position space:
\begin{align}
\psi_n(x,t) &= e^{-i (n + 1/2) \omega t} \psi_n(x).
\end{align}

\section{General State and Time Evolution}

Any state can be expanded in the energy basis:
\begin{align}
\ket{\psi(0)} = \sum_{n=0}^\infty c_n \ket{n}, \qquad \sum_{n=0}^\infty \abs{c_n}^2 = 1.
\end{align}
The time-evolved state is
\begin{align}
\ket{\psi(t)} &= \sum_{n=0}^\infty c_n e^{-i (n + 1/2) \omega t} \ket{n}.
\end{align}
The position-space wave function is therefore
\begin{align}
\psi(x,t) &= \sum_{n=0}^\infty c_n \, e^{-i (n + 1/2) \omega t} \, \psi_n(x).
\label{eq:general_time_evolution}
\end{align}

If the expansion is truncated to a finite number of terms (as done numerically), one simply sums over the included $n$ and renormalizes if necessary:
\begin{align}
\psi(x,t) \approx \frac{\sum_{n=0}^{N} c_n e^{-i (n + 1/2) \omega t} \psi_n(x)}
{\sqrt{\sum_{n=0}^{N} \abs{c_n}^2}}.
\end{align}


\section{Summary of Formulas Used in the Python Code}

The code you posted implements exactly Eq.~\eqref{eq:general_time_evolution} with coefficients
\begin{align}
c_0 = 1, \quad c_1 = 0, \quad c_2 = 3, \quad c_3 = 7
\end{align}
(then normalized inside the function), $x_0 = \sqrt{\hbar/(m\omega)}$, and time evolution factor $e^{-i(n+1/2)\omega t}$ for each term.

The probability density $\abs{\psi(x,t)}^2$ shows a complicated but periodic motion with period $T = 2\pi/\omega$ (revival time, since all phases are multiples of $\omega t$).

\end{document}
